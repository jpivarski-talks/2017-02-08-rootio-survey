\documentclass{beamer}

\mode<presentation>
{
  \usetheme{default}
  \usecolortheme{default}
  \usefonttheme{default}
  \setbeamertemplate{navigation symbols}{}
  \setbeamertemplate{caption}[numbered]
  \setbeamertemplate{footline}[page number]
  \setbeamercolor{frametitle}{fg=white}
  \setbeamercolor{footline}{fg=black}
} 

\usepackage[english]{babel}
\usepackage[utf8x]{inputenc}
\usepackage{tikz}
\usepackage{listings}
\usepackage{courier}
\usepackage{array}
\usepackage{bold-extra}
\usepackage{minted}

\xdefinecolor{darkblue}{rgb}{0.1,0.1,0.7}
\xdefinecolor{darkgreen}{rgb}{0,0.5,0}
\xdefinecolor{darkgrey}{rgb}{0.35,0.35,0.35}
\xdefinecolor{darkorange}{rgb}{0.8,0.5,0}
\xdefinecolor{darkred}{rgb}{0.7,0,0}
\xdefinecolor{dianablue}{rgb}{0.18,0.24,0.31}
\definecolor{commentgreen}{rgb}{0,0.6,0}
\definecolor{stringmauve}{rgb}{0.58,0,0.82}

\lstset{ %
  backgroundcolor=\color{white},      % choose the background color
  basicstyle=\ttfamily\small,         % size of fonts used for the code
  breaklines=true,                    % automatic line breaking only at whitespace
  captionpos=b,                       % sets the caption-position to bottom
  commentstyle=\color{commentgreen},  % comment style
  escapeinside={\%*}{*)},             % if you want to add LaTeX within your code
  keywordstyle=\color{blue},          % keyword style
  stringstyle=\color{stringmauve},    % string literal style
  showstringspaces=false,
  showlines=true
}

\lstdefinelanguage{scala}{
  morekeywords={abstract,case,catch,class,def,%
    do,else,extends,false,final,finally,%
    for,if,implicit,import,match,mixin,%
    new,null,object,override,package,%
    private,protected,requires,return,sealed,%
    super,this,throw,trait,true,try,%
    type,val,var,while,with,yield},
  otherkeywords={=>,<-,<\%,<:,>:,\#,@},
  sensitive=true,
  morecomment=[l]{//},
  morecomment=[n]{/*}{*/},
  morestring=[b]",
  morestring=[b]',
  morestring=[b]"""
}

\title[2017-02-08-rootio-survey]{Survey of columnar file formats and the techniques they use}
\author{Jim Pivarski}
\institute{Princeton -- DIANA}
\date{February 8, 2017}

\begin{document}

\logo{\pgfputat{\pgfxy(0.11, 8)}{\pgfbox[right,base]{\tikz{\filldraw[fill=dianablue, draw=none] (0 cm, 0 cm) rectangle (50 cm, 1 cm);}}}\pgfputat{\pgfxy(0.11, -0.6)}{\pgfbox[right,base]{\tikz{\filldraw[fill=dianablue, draw=none] (0 cm, 0 cm) rectangle (50 cm, 1 cm);}\includegraphics[height=0.99 cm]{diana-hep-logo.png}\tikz{\filldraw[fill=dianablue, draw=none] (0 cm, 0 cm) rectangle (4.9 cm, 1 cm);}}}}

\begin{frame}
  \titlepage
\end{frame}

\logo{\pgfputat{\pgfxy(0.11, 8)}{\pgfbox[right,base]{\tikz{\filldraw[fill=dianablue, draw=none] (0 cm, 0 cm) rectangle (50 cm, 1 cm);}\includegraphics[height=1 cm]{diana-hep-logo.png}}}}

% Uncomment these lines for an automatically generated outline.
%\begin{frame}{Outline}
%  \tableofcontents
%\end{frame}

%%%%%%%%%%%%%%%%%%%%%%%%%%%%%%%%%%%%%%%%%%%%%%%%%%%%%%%

\begin{frame}{The ROOT file format is many things}
\begin{itemize}\setlength{\itemsep}{0.2 cm}
\item \only<1,3>{Generic}\only<2>{\textcolor{blue}{\underline{Generic}}}: designed for any type of data.

\item Key-value object store for histograms, lookup tables, etc.

\item \only<1,3>{Big data storage}\only<2>{\textcolor{blue}{\underline{Big data storage}}}: identically structured files with large TTrees.

(Byte for byte, this is by far the most common use-case!)

\item \only<1,3>{Binary and schemaed}\only<2>{\textcolor{blue}{\underline{Binary and schemaed}}} (TStreamerInfo) for efficient access.

\item \only<1>{Hierarchical data}\only<2,3>{\textcolor{blue}{\underline{Hierarchical data}}}, such as events containing jets containing tracks containing hits.

\item Remotely accessible via the XRootD protocol.

\item Record-oriented or \only<1>{columnar}\only<2,3>{\textcolor{blue}{\underline{columnar}}} with a configurable splitting level.

\item And now \only<1,3>{multilingual}\only<2>{\textcolor{blue}{\underline{multilingual}}}, with jsROOT, root4j, and soon go-root.
\end{itemize}

\begin{uncoverenv}<2,3>
\vspace{0.3 cm}
This talk will be about file formats that share \only<2>{\textcolor{blue}{these features}}\only<3>{these features} and what we can learn from them. \only<3>{\textcolor{blue}{But especially these.}}
\end{uncoverenv}
\end{frame}

\begin{frame}{\underline{Columnar} data}
\vspace{0.5 cm}
ROOT's TNtuples resemble database storage formats: user usually only touches a subset of database columns, so it's important to access a few columns without being slowed down by the others.

\vspace{0.3 cm}
\textcolor{darkblue}{Example:} ORC file format for Hive (Hadoop as a database). Each data column is saved as a contiguous, equal-length array.

\vspace{0.75 cm}
\textcolor{darkgray}{Generic, binary, non-columnar formats, such as ProtocolBuffers, Thrift, and Avro, are better suited to remote procedure calls (RPC) and streaming analytics (``live'' data without storage).}
\end{frame}

\begin{frame}{\underline{Hierarchical} data}
\vspace{0.5 cm}
Although SQL-99 introduced arrays and structures, the support is underwhelming.

\vspace{0.3 cm}
\textcolor{darkgray}{(For instance, how would you pick out the {\tt px}, {\tt py}, {\tt pz} of the top two muons in an event and construct an invariant mass in SQL?)}

\vspace{0.3 cm}
ORC files store arrays and structures within a column ``unsplit.'' If you want one subfield, you have to load or skip over all subfields.

\begin{uncoverenv}<2>
\vspace{0.5 cm}
Nevertheless, the industry is moving in this direction: a Google paper (\href{https://research.google.com/pubs/pub36632.html}{\textcolor{blue}{\underline{link}}}) described a hierarchical, columnar file format, used in-house since 2006.

\vspace{0.3 cm}
It became the basis for Apache Parquet (file format), Apache Arrow (in-memory data representation), SparkSQL 2.0 optimizations, Ibis, Impala, Kudu, Drill query servers, and probably others.
\end{uncoverenv}
\end{frame}

\begin{frame}{They didn't know about ROOT}
\vspace{0.5 cm}
\mbox{ } \hfill \textcolor{darkblue}{\large Dremel: Interactive Analysis of Web-Scale Datasets (2010)} \hfill \mbox{ }

\vspace{0.5 cm}
\includegraphics[width=\linewidth]{dremel.png}

\vspace{0.5 cm}
\mbox{ } \hfill \textcolor{darkblue}{\Large Independently developed: this is xenobiology!} \hfill \mbox{ }
\end{frame}

\begin{frame}{Repetition and definition levels}
\vspace{0.5 cm}
\begin{columns}[T]
\column{0.4\linewidth}
\includegraphics[width=\linewidth]{repetition_and_definition_schema.png}

\column{0.5\linewidth}
\includegraphics[width=\linewidth]{repetition_and_definition_data.png}
\end{columns}

\vspace{0.5 cm}
\includegraphics[width=\linewidth]{repetition_and_definition_arrays.png}
\end{frame}


%% \begin{frame}{Binary, schemaed, \underline{columnar} data}
%% \vspace{0.5 cm}
%% \ldots\ in other words, a database storage format.

%% \vspace{0.5 cm}
%% User queries often touch only a subset of database columns, so it's important to be able to access a few columns without being slowed down by the others.

%% \vspace{0.3 cm}
%% \textcolor{darkblue}{Example:} ORC file format for Hive (Hadoop as a database). Each data column is saved as a contiguous, equal-length array.

%% \begin{uncoverenv}<2>
%% \vspace{0.5 cm}
%% \textcolor{darkblue}{This is a specialization.} In remote procedure calls (RPC) and streaming analytics (``live'' data without storage, such as Twitter analysis and maybe online monitoring in HEP), it's advantageous to make each record contiguous.

%% \vspace{0.3 cm}
%% \textcolor{darkblue}{Examples:} Protobuf, Thrift, Avro.
%% \end{uncoverenv}
%% \end{frame}

%% \begin{frame}{\underline{Hierarchical} columnar data}

%% Databases usually store flat ntuples: SQL tables whose columns are scalars: numbers, booleans, or strings.

%% ORC stores SQL-99 features like record structures and arrays in record-major 

%% \end{frame}



\end{document}
